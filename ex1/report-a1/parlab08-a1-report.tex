\documentclass[a4paper,10pt]{article} \usepackage{anysize}
\marginsize{2cm}{2cm}{1cm}{1cm}
%\textwidth 6.0in \textheight = 664pt
\usepackage{xltxtra}
\usepackage{xunicode}
\usepackage{graphicx}
\usepackage{color}
\usepackage[table]{xcolor}
%\usepackage[usenames,dvipsnames]{xcolor}
\usepackage{xgreek}
\usepackage{fancyvrb}
\usepackage{minted}
\usepackage{listings}
\usepackage{enumitem} \usepackage{framed} \usepackage{relsize}
\usepackage{float} 
\setmainfont[Mapping=TeX-text]{CMU Concrete}


\begin{document}

\begin{titlepage}
\begin{center}
\begin{figure}[t] 
     \includegraphics[scale=0.7]{title/ntua_logo}
\end{figure}
\begin{LARGE}\textbf{ΕΘΝΙΚΟ ΜΕΤΣΟΒΙΟ ΠΟΛΥΤΕΧΝΕΙΟ\\}\end{LARGE}
\vspace{2cm}
\begin{Large}
ΣΧΟΛΗ ΗΜ\&ΜΥ\\
Συστήματα Παράλληλης Επεξεργασίας
3\textsuperscript{η} Άσκηση\\
Ακ. έτος 2012-2013\\
\end{Large}
\vspace{5cm}
\Large Ομάδα 8\textsuperscript{η}\\
\vspace{1cm}
\begin{tabular}{l r}
\Large{Μαυρογιάννης Αλέξανδρος}&
\large{Α.Μ.: 03109677}\\
\Large{Λύρας Γρηγόρης}&
\large{Α.Μ.: 03109687}\\
\end{tabular}\\
\vspace{5cm}

\vfill
\large\today\\
\end{center}
\end{titlepage}





\setcounter{section}{1}

%{{{ OpenMP

\section{Μοντέλο κοινού χώρου διευθύνσεων (\textbf{OpenMP})}
Παρακάτω ακολουθούν τα διαγράμματα speedup-threads των εκτελέσεων OpenMP.
Παρατηρούμε πως ενώ η καμπύλη είναι σχεδόν γραμμική για μέγεθος εισόδου 1024,
σε μέγεθος εισόδου 2048 το μέγιστο speedup πέφτει κάτω απο το 2. Αυτό οφείλεται
στο οτι ο πίνακας πλέον δεν χωράει στην cache και δεν εκμεταλεύται όπως πριν το
locality.

\begin{figure}[H]
    \centering
    \includegraphics[width=\textwidth]{files/src-outparse_omp-512_speedup.png}
    \caption{512 speedup}
    \label{fig:omp_512_speed} \end{figure}

\begin{figure}[H]
    \centering
    \includegraphics[width=\textwidth]{files/src-outparse_omp-1024_speedup.png}
    \caption{1024 speedup}
    \label{fig:omp_1024_speed}
\end{figure}

\begin{figure}[H]
    \centering
    \includegraphics[width=\textwidth]{files/src-outparse_omp-2048_speedup.png}
    \caption{2048 speedup}
    \label{fig:omp_2048_speed}
\end{figure}



%}}} OpenMP

\pagebreak

%{{{ MPI
\section{Μοντέλο ανταλλαγής μηνυμάτων (\textbf{MPI})}

\subsection{Speedups}

Στα ακόλουθα σχήματα επιτάχυνσης παραθέτουμε τα speedups που παρατηρήσαμε στις
εκτελέσεις για τα διάφορα μεγέθη πίνακα για κάθε έκδοση κώδικα, ως προς τον
αριθμό των MPI διεργασιών που εκτελούνταν.

Στο σχήμα \ref{fig:4096_speed} παρατηρούμε μέγιστο speedup καθώς μετά απο αυτή
την εκτέλεση το μέγεθος του πίνακα που μοιράζουμε γίνεται μικρότερο από το
cache size.

\begin{figure}[H]
    \centering
    \includegraphics[width=0.85\textwidth]{files/src-outparse-2048_speedups.png}
    \caption{2048 speedups}
    \label{fig:2048_speed}
\end{figure}

\begin{figure}[H]
    \centering
    \includegraphics[width=0.85\textwidth]{files/src-outparse-4096_speedups.png}
    \caption{4096 speedups}
    \label{fig:4096_speed}
\end{figure}

\begin{figure}[H]
    \centering
    \includegraphics[width=0.85\textwidth]{files/src-outparse-6144_speedups.png}
    \caption{6144 speedups}
    \label{fig:6144_speed}
\end{figure}




\subsection{Χρόνοι εκτέλεσης}

Στα ακόλουθα διαγράμματα με μπάρες  παραθέτουμε τους χρόνους εκτέλεσης και τους
χρόνους επικοινωνίας για τα διάφορα μεγέθη πίνακα σε κάθε έκδοση κώδικα.
Παρατηρούμε πως οι cyclin (round robin) υλοποιήσεις απαιτούν περισσότερο
computation time απο τις continuous, αλλα έχουν σημαντικά μικρότερο
communication time.

Συνολικά, για μικρό αριθμό διεργασιών βλέπουμε πως οι cyclic είναι πιο γρήγορες,
αλλά αυτή η εικόνα αλλάζει όσο αυξάνονται οι διεργασίες αφού οι cyclic απαιτούν
περισσότερο communication time ενώ οι continous λιγότερο. 

Τέλος, παρατηρούμε πως οι point-to-point εκδοχές είναι πιο αργές στο
communication time και έχουν χειρότερο scaling απο τις αντίστοιχες collective. 

\begin{figure}[H]
    \centering
    \includegraphics[width=\textwidth]{files/src-outparse-2048_times.png}
    \caption{2048 times}
    \label{fig:2048_time}
\end{figure}

\begin{figure}[H]
    \centering
    \includegraphics[width=\textwidth]{files/src-outparse-4096_times.png}
    \caption{4096 times}
    \label{fig:4096_time}
\end{figure}

\begin{figure}[H]
    \centering
    \includegraphics[width=\textwidth]{files/src-outparse-6144_times.png}
    \caption{6144 times}
    \label{fig:6144_time}
\end{figure}

%}}} MPI

\end{document}



