\documentclass[a4paper,10pt]{article} \usepackage{anysize}
\marginsize{2cm}{2cm}{1cm}{1cm}
%\textwidth 6.0in \textheight = 664pt
\usepackage{xltxtra}
\usepackage{xunicode}
\usepackage{graphicx}
\usepackage{color}
\usepackage[table]{xcolor}
%\usepackage[usenames,dvipsnames]{xcolor}
\usepackage{xgreek}
\usepackage{fancyvrb}
\usepackage{minted}
\usepackage{listings}
\usepackage{enumitem} \usepackage{framed} \usepackage{relsize}
\usepackage{float} 
\setmainfont[Mapping=TeX-text]{CMU Concrete}


\begin{document}

\begin{titlepage}
\begin{center}
\begin{figure}[t] 
     \includegraphics[scale=0.7]{title/ntua_logo}
\end{figure}
\begin{LARGE}\textbf{ΕΘΝΙΚΟ ΜΕΤΣΟΒΙΟ ΠΟΛΥΤΕΧΝΕΙΟ\\}\end{LARGE}
\vspace{2cm}
\begin{Large}
ΣΧΟΛΗ ΗΜ\&ΜΥ\\
Συστήματα Παράλληλης Επεξεργασίας
3\textsuperscript{η} Άσκηση\\
Ακ. έτος 2012-2013\\
\end{Large}
\vspace{5cm}
\Large Ομάδα 8\textsuperscript{η}\\
\vspace{1cm}
\begin{tabular}{l r}
\Large{Μαυρογιάννης Αλέξανδρος}&
\large{Α.Μ.: 03109677}\\
\Large{Λύρας Γρηγόρης}&
\large{Α.Μ.: 03109687}\\
\end{tabular}\\
\vspace{5cm}

\vfill
\large\today\\
\end{center}
\end{titlepage}





\setcounter{section}{1}

%{{{ Intro
\section{Αλγόριθμος DMV}
Ο αλγόριθμος πολλαπλασιασμού πίνακα με διάνυσμα (Dence Matrix-Vector
multiplication) είναι από τους συνιθέστερους υπολογιστικούς πυρήνες αλγεβρικών
υπολογισμών. Αυτός συνοψίζεται στην παρακάτω σχέση.

\[
    y_i = \sum_{j=1}^{N}a_{ij}\times x_{j}, \forall i \in [1,N]
\]

%}}}


\section{Υλοποιήσεις σε GPU}

\subsection{Naive kernel}
Κάθε πρόσβαση σε δεδομένα γίνεται στο κομμάτι της shared μνήμης. Αυτή έχει πιο
αργή πρόσβαση και κατά συνέπεια το πρόβλημά μας βρίσκεται στο εύρος ζώνης του
διαδρόμου μνήμης.

\subsection{Coalesced kernel}
Σε αυτή την έκδοση τροποποιήσαμε τον kernel ώστε κάθε group από threads της
GPU να υπολογίζει μία γραμμή του τελικού πίνακα. Με αυτή την τροποποίηση
χρησιμοποιούμε την τοπική μνήμη για να αποθηκεύσει ένα μέρος του αποτελέσματος
στη μεταβλητή partial product, όπου κάθε thread τοποθετεί ατομικά το μέρος του
αθροίσματος που υπολογίζει. Τέλος γίνεται μια λογαριθμική μείωση παράλληλα από
τα threads ώστε να καταλήξουμε στο τελικό αποτέλεσμα που περνιέται στην
αντίστοιχη σειρά του $y$.


\subsection{Local memory kernel}
Στην αρχιτεκτονική Fermi η local memory χρησιμοποιείται ως cache συνεπώς η
προσπάθεια που καταβάλαμε για την χρήση της τοπικής μνήμης δεν έχει εμφανές
αποτέλεσμα.


\end{document}



